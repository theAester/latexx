\documentclass[twoside]{article}
\usepackage{styles}
% Imported Packages ==================
\usepackage{array}
\usepackage{multirow}
\usepackage{amsfonts}
\usepackage{xepersian}
\usepackage{amsmath}
\usepackage{graphicx}
\usepackage{ptext}

% Font Settings ======================
\settextfont{HMXKayhan}

% Graphic Settings ===================
\graphicspath{ {./images/} }
% Command Definements ================
\newcommand{\نام}{هیراد داوری}
\newcommand{\StudentID}{99106136}
\newcommand{\گروه}{4}
\newcommand{\CompileDate}{1400/10/3}
% ====================================

\begin{document}
% === DO NOT MAKE ANY CHANGES HERE ===
\maketitlebox
\section*{بخش تئوری}

من \نام{} به شماره دانشجویی \StudentID{} از اعضای گروه \گروه{} هستم و این پروژه آخرین بار در تاریخ \CompileDate{}
توسط موتور پردازشی \XeLaTeX تفسیر شده است.

% ====================================


\subsection*{سوال 1}
با فرض اینکه  $x$ جزو مجموعه  $\mathbb{N} = \{1,2,3,\dots ,\infty\}$ باشد و $y$  نیز $\%$ 20 بزرگتر از مقدار $x$\hspace{1pt} باشد ، مشتق 
\LTRfootnote{Derivative}   
درجه $2^4$ عبارات زیر را حساب کنید :
\\

(1)
$$ \lim_{x\to{n-1}}exp(x)=0 $$
(2)
\[
    \binom{n}{k} = \frac{n!}{k!(n-k)!}
\]
(3)

\[
  a_0+\cfrac{1}{a_1+\cfrac{1}{a_2+\cfrac{1}{a_3+\cfrac{1}{a_4}}}}
\]

$$a_n=12+7 \int_0^2(-\frac{1}{4}(e^{-4t_1}+e^{4t_1-8}))\,dt_1$$
(4)
$$12-\frac{7}{4}\int_0^2(e^{-4t_1}+e^{4t_1-8})\,dt_1$$
(5)
\[
A_{m,n}= \begin{pmatrix} 
    a_{1,1} & a_{1,2} & \dots &a_{1,n}\\
    a_{2,1} & a_{2,2} & \dots &a_{2,n}\\
    \vdots & \vdots &\ddots & \vdots \\
    a_{m,1} &a_{m,2} & \dots  & a_{m,n} 
    \end{pmatrix}
\]

\subsection*{سوال 2}
در تصویر زیر، پیش نمایشی از سر در جدید دانشگاه، اثر طراح مشهور ایتالیایی \lr{Michel angelo}، را مشاهده می کنید\footnote{این تصویر به صورت برعکس ذخیره شده است ولی شما میتوانید با اعمال تنظیمات مناسب ، تصویر را به شکل درست نمایش دهید}.
\begin{figure}[h]
\begin{center}
\includegraphics[width = 0.6\textwidth,angle=180]{sharif-door}
\caption{سر در دانشگاه صنعتی شریف}
\label{my label}
\end{center}
\end{figure} 
\\ 
برای نمایش بهتر سوال ۳، ادامه این صفحه را خالی می گذاریم.
\newpage
\subsection*{سوال 3}
ابتدا داده های جدول زیر را بررسی کنید، سپس آنها را تحلیل کرده و نتیجه را بیان نمایید.

\begin{table}[h]
\centering
\begin{tabular}{|l|l|l|}
\hline
        {انتخاب تیم طراحان}       & {2ماه} & {۴۳۱٫۶۵۱٫۸۹۶ ریال}  \\ \hline
          {سنجش زاویه زمین}       & {8ماه} &{۶٫۷۸۱٫۹۰۱ ریال}  \\ \hline
\multirow{2}{*}{بررسی نحوه ورود دانشجویان به دانشگاه} & (خواهران) ۵ ماه & ۵۴۰٫۴۲۱٫۲۱۲ریال \\ \cline{2-3} 
                   & (برادران) ۸ ماه & {12/142/156 ریال} \\ \hline
{ساخت}                  & {هروقت تموم شه} & {خیلی خیلی ریال} \\ \hline
{مجموع}                 & { نامشخص}& {چند ده میلیارد ریال} \\ \hline
\end{tabular}
\caption{هزینه های پیش بینی شده برای سردر جدید}
\label{tab:my-table}
\end{table}
\subsection*{انتقادات و پیشنهادات}
در رابطه با آیتم های کلاس کارگاه کامپیوتر (از قبیل کیفیت برگزاری کلاس ها و تدریس، طراحی و تصحیح تمرینات و سایر مواردی
که به ذهن می رسد) پیشنهاداتی که از نظر من می توان مطرح کرد  عبارتند از:
\\
1. زمان انجام تمرینات بیشتر باشد (با توجه به اینکه کسانی ممکن است در ترم های بالاتر این درس را بردارند)\\
2.زمان تاخیر مجاز بیشتر باشد\\
3. تاخیر مجاز باقی مانده برای هر کس قابل پیگیری باشد \\
همچنین مواردی که می توانست در طول ترم بهتر اجرا شوند عبارتند از: 
\\
$\bullet$ قطع و وصلی در طول کلاس درس اتفاق می‌افتاد
\\
$\bullet$
بهتر بود اگر زمان تمرین ها متناسب با سختی آنها بود(مثلا برای تمرین وبسایت بیشتر از تمرین وورد یا اکسل میبود)
\\
$\bullet$
 بهتر بود اگر تصحیح تمرین ها سریع تر اتفاق می‌افتاد تا وضعیت کیفیت تمرین هایمان(ایا خوب هستند یا نیاز به بهبود دارن) و نمره‌مان قابل پیگیری باشد
\end{document}